\title{FZ Theory: Theory of Nothing}

\author{DarwinUroborovich}

\date{2025} 

\begin{document}

\maketitle

\begin{abstract}
FZ Theory (also known as the Theory of Nothing) is a conceptual framework explaining how existence could emerge from a state of absolute nothingness, given infinite ``time of nothingness'' and an unbounded set of possibilities. It postulates that infinite nothingness, combined with a non-metric infinite process of trials, will inevitably produce every possible configuration of reality at least once. Among these random fluctuations, only stable worlds (self-consistent systems of physical laws and structures) persist, while unstable configurations collapse back into nonexistence. Crucially, FZ Theory asserts the necessity of a conscious observer arising simultaneously with a stable world, since consciousness is required to define and perceive that world’s boundaries. This initial world and its accompanying consciousness form a kind of primordial being or framework, within which subsequent nested realities can develop—each new level acquiring a more concrete structure (a process termed ``condensation'' of reality). This work lays out the axiomatic foundations of FZ Theory, describes the mechanism of world formation from nothing, and presents results of computer simulations (six unit tests) that support the internal consistency of the theory. The theory is analyzed for logical coherence and compared with known scientific principles, demonstrating no contradictions with established physical laws while offering a novel perspective on cosmogenesis.
\end{abstract}

\section{Introduction}
The question of how something can arise from nothing has preoccupied philosophers and scientists for centuries. Traditional cosmology usually assumes an initial singularity or a quantum fluctuation; however, FZ Theory offers a different approach by postulating an initial state of absolute nothingness existing for an infinitely long duration. Instead of invoking pre-existing laws or a divine creator, this theory suggests that given an infinitely long ``time of nothingness,'' even the most improbable events (including the spontaneous emergence of a universe) will eventually occur. In an infinite void, where ``nothingness'' has no physical properties (zero energy, no matter, no space or time), its very boundlessness over ``time'' provides the arena for random occurrences.

FZ Theory rests on a set of fundamental assumptions (axioms) and leads to a scenario in which the Universe (and possibly a multitude of universes) can emerge spontaneously from nonexistence. A key element is the role of consciousness: according to the theory, for a world to realize (to attain definite existence and boundaries), a form of consciousness capable of observing it must emerge at the same time. This approach inextricably links the appearance of the physical world with the appearance of an observer, implying that reality and consciousness are two sides of the same creative act.

In the following sections, the axiomatic foundations of the theory are outlined, the process by which a stable world arises from nothing is described, and the simultaneous emergence of consciousness is examined. The concept of nested creation is introduced (formation of worlds within other worlds with increasing ``density'' of reality). Next, results are summarized from a series of computer experiments (unit tests) that simulate key aspects of the theory and demonstrate consistency with its predictions. Finally, the correspondence of FZ Theory with established scientific notions and its distinctions are discussed, highlighting the novel insight it brings to cosmological thought.

\section*{Axiomatic Foundation}

FZ Theory is based on a set of fundamental axioms defining the properties of ``nothingness'' and the conditions for transitioning to ``something.'' These axioms form the logical framework of the theory:

\begin{enumerate}
    \item \textbf{Infinite nothingness.} It is postulated that there exists an infinite void of absolute nothingness (no matter, energy, or curvature of spacetime), enduring for an infinitely long duration. Here, ``infinitely long'' does not mean ordinary linear time, but an abstract non-metric process — a kind of infinite sequence of independent attempts (``time of nothingness''). In other words, absolute nothingness serves as a hyper-infinite reservoir of potential (formally $\Phi \to \infty$), where the absence of any boundaries guarantees that even the most incredibly rare configuration will eventually be realized at least once. By virtue of an infinite number of such attempts, the probability of the appearance of any arbitrarily unlikely self-consistent configuration ultimately becomes 1 (the inevitability of realization given unlimited trials). Despite the absence of any content, this nothingness provides an unbounded background for potential events.

    \item \textbf{Unlimited possibilities.} In infinite nothingness, any conceivable configuration of physical laws or ``proto-universe'' can potentially arise, given enough attempts and ``time of nothingness.'' Even if the probability of spontaneously forming a particular configuration is extraordinarily small, the infinity of time and space makes the eventual appearance of each possibility inevitable (effectively, the probability of realization becomes 1 with an unbounded number of independent trials). In other words, nothingness cycles through all possible modes of being, however many there may be.

    \item \textbf{Selection by stability.} Each time a proto-universe or configuration arises in nothingness, by its internal laws and structure it is either initially stable or unstable. Stable worlds (i.e., those that possess self-consistent laws and do not immediately lead to contradictions or collapse) can persist for a significant duration, whereas unstable worlds soon after formation disintegrate back into nothingness. Thus, only stable configurations survive any meaningful interval of ``time'' and manifest in some way, whereas unsound configurations disappear without a trace. The infinite nothingness acts here as a natural filter, selecting robust worlds from an innumerable multitude of trials.

    \item \textbf{Simultaneous emergence of consciousness.} For any stable world to fully exist, some form of consciousness must arise along with it. This axiom states that to observe and define the boundaries of a world, the presence of an observer is necessary. Without a conscious observer, even a stable configuration remains an indeterminate abstraction. Thus, the first stable world is accompanied upon its emergence by at least one conscious being capable of perceiving it and registering its characteristics.

    \item \textbf{Primary observer.} The initial consciousness that arises together with the first stable world constitutes the primary observer — a being or mind that effectively ``collapses'' the indeterminate potential into a definite reality. Informally, this can be likened to a divine being, but here it is a natural outcome of the selection process rather than a presupposed supernatural premise. This observer provides a frame of reference that gives the new world concrete existence and serves as a guarantor of its definiteness.

    \item \textbf{Nested creation (condensation).} After the emergence of the first stable world and its associated consciousness, additional worlds or layers of reality can form within this initial creation. Each subsequent ``generation'' of worlds undergoes a process of condensation, acquiring more concrete and tightly defined laws and boundaries. In other words, once the primordial universe (with its base laws and observer) has appeared, new, more ``dense'' realities — essentially sub-universes or nested worlds — can originate within it. Each next level has increased structuredness, while still obeying the constraints and laws of the higher-level world. This iterative cosmogony leads to a hierarchy of realities, where the uppermost level — the initial world — is the most abstract, and the lower levels become progressively more concrete.

    \item \textbf{Lawlike emergence.} Although the initial formation of a world from nothing occurs through random trials, the resulting stable world is governed by coherent physical laws. These laws are ``chosen'' from the space of possibilities: what may appear as a chaotic beginning yields an outcome in which consistent rules operate at micro- and macroscopic scales. In other words, the theory holds that even an apparently chaotic genesis still conforms to inner logic — any world that emerges must adhere to its own self-consistent principles (for example, conservation laws, causality, etc.) as soon as it takes shape. Otherwise, the world simply will not survive (see Axiom 3). Thus, worlds that pass the selection filter will later appear orderly to their inhabitants, even though their initial parameters may have been randomly determined.
\end{enumerate}

Taken together, these axioms describe a scenario in which something is born from nothing — and it occurs in a certain order: randomness provides the exploration of possibilities, stability acts as the filter, and consciousness gives reality its definiteness. The resulting framework of the theory is characterized by internal unity: the infinite nothingness serves as the engine of creation, ensuring the inevitability of a world’s emergence, while the remaining axioms define the properties and structure of that emergent world.

\section*{Emergence of a Stable World from Nothingness}

According to FZ Theory, the birth of a universe (a stable world) is not a one-time miracle, but an event that sooner or later must happen given an infinite number of attempts. One can imagine infinite nothingness as a boundless sea of zero reality, in which random fluctuations or ``attempts'' to form something occur sporadically in space and in non-metric time:

\begin{itemize}
    \item \textbf{Inevitability of fluctuations:} Because the ``time'' of nothingness is infinite and the nothingness itself has no boundaries, even extremely unlikely fluctuations will one day lead to the emergence of exactly that combination of conditions which can form a coherent world or a complete system of physical laws. This can be likened to how, given an infinite number of attempts, a working mechanism might eventually assemble itself from scattered parts. In other words, statistically, with an infinite number of independent ``chances,'' even an event of vanishingly small probability will in the end occur with near certainty (see the Appendix for a mathematical demonstration of this principle).

    \item \textbf{Multiplicity of attempts:} Such genesis of worlds may happen not just once but many times, in different regions of the void. In other words, within infinite nothingness, numerous proto-worlds can originate independently of each other. Most of these attempts will be short-lived (unstable), but some may meet the criteria of stability and endure. Thus, nothingness ``tries out'' an infinite number of configurations both simultaneously and sequentially, and successful cases can arise in different locations of the boundless non-being.

    \item \textbf{Fate of unstable configurations:} An unstable proto-universe is a configuration that, for example, violates some fundamental principle of consistency (say, it contains internal contradictions in its laws or impossible states). Such a configuration is essentially ``rejected'' by nothingness and almost immediately disintegrates, reverting to the initial state of nonexistence. Each failed attempt produces zero lasting effects: a world flickers into existence and then extinguishes, leaving no trace, like an instantaneous quantum vacuum fluctuation.

    \item \textbf{Survival of a stable world:} A stable proto-universe, by contrast, possesses a self-sustaining structure. It contains a non-contradictory set of physical laws (selected from the countless possibilities) that allows it to hold itself in being. Once such a stable world arises, it does not vanish instantly — it develops internal structure and the potential for long-term existence, independent of the surrounding nothingness. Figuratively speaking, nothingness ``finds'' a rare working combination, and from that moment the new something acquires an independent existence.
\end{itemize}

The process of forming a world from nothing can be viewed as a gigantic search through all possible realities: the infinite nothingness, having unlimited attempts, eventually discovers a configuration that works — that is, yields a cosmos capable of persisting. All unstable (unsuccessful) trials leave no trace, whereas a successful stable result marks the transition from non-being to being. This scenario resolves the apparent paradox of the ``impossibility'' of something emerging from absolute zero: an infinite number of zero outcomes can sum to a nonzero result, because although each individual fluctuation almost surely will not create a world, sooner or later one of them will. As soon as a stable world appears, it is no longer ``nothing'' — it is something, born from the collective potential of nothingness, and it begins to exist according to its own laws.

It is important to emphasize that the appearance of a stable world is a deterministic outcome of undetermined trials. Probabilistic chaos at the pre-physical stage eventually gives rise to an ordered Universe. For example, if our Universe is indeed such a stable world, then its laws (physical constants, interactions, etc.) are the result of a ``winning ticket'' in the infinite lottery of nothingness. Despite the random genesis, from the moment of its formation the Universe behaves lawfully and predictably, obeying its internal rules. The configuration that passed the selection no longer changes at random, but instead evolves within the established regularities.

\section*{The Role of Consciousness}

One of the central tenets of FZ Theory is the inseparability of the appearance of the physical world and consciousness. The role of consciousness in this theory sheds light on an age-old philosophical question: if a universe formed but there were no consciousness to observe it, in what sense does it ``exist'' or have defined properties?

According to FZ Theory, at the very moment a stable world arises, at least one conscious being manifests along with it. This is not a divine act of creation, but a natural pairing of world and observer dictated by the necessity of perception. The first stable world and the first consciousness are two facets of the single event of reality’s genesis:

\begin{itemize}
    \item \textbf{Definition of reality’s boundaries:} Consciousness is necessary for recognizing and defining the boundaries and content of a new world. Without an observer, concepts like space, time, and matter have neither reference point nor concrete meaning. A conscious observer provides the context in which the characteristics of the world can be discerned, measured, and fixed. In other words, consciousness ``illuminates'' reality, making it an established fact.

    \item \textbf{Primary conscious being:} The initial consciousness can be envisioned as the primary observer or primordial mind, possessing the point of view from which the world is observed for the first time. In theological terms, such an entity might be likened to God, but within this theory it arises naturally as a consequence of the cosmic selection process rather than being an external supernatural premise. This observer does not exist prior to the Universe’s appearance; rather, it comes into being simultaneously with the Universe as its intrinsic witness. Importantly, the primary mind need not be complex or human-like — it could be the most rudimentary level of awareness, as long as it fulfills the function of observing the new world. The main point is that the world does not remain without a witness to its existence.

    \item \textbf{Mutual realization:} The Universe and consciousness mutually define each other. The world provides the stage for the observer, and the observer, in turn, ``collapses'' the world from a state of potentiality into a definite reality by perceiving it. In essence, the act of observation is the precondition for the very existence of reality as a determinate system: no world can stably form or acquire concrete features without a subject that registers its presence. In quantum analogy terms, one might say that a world without an observer remains a superposition of possibilities, and only with the advent of consciousness does an irreversible selection of the reality’s parameters occur.
\end{itemize}

By asserting the necessity of a conscious presence at the moment of creation, FZ Theory weaves subjective experience into the very fabric of cosmogenesis. This approach challenges the usual separation between physical reality and consciousness, suggesting instead that they emerge hand in hand. It offers a possible explanation for why conscious beings (like humans) eventually appeared in the Universe: not as a random byproduct of matter’s development, but as an initially inherent feature of the first act of being. In the subsequent evolution of the Universe, consciousness could become more complex and take on many forms (which we indeed observe as the diversity of life and mind), but all of them, according to this concept, trace back to that original spark of awareness that accompanied the birth of the world.

\section*{Nested Realities and Condensation}

After the formation of the first stable world (the primordial Universe with its primary observer), FZ Theory allows for the possibility of further creative processes within that world. This concept is described as the emergence of nested realities and the gradual condensation of existence:

\begin{itemize}
    \item \textbf{Nested worlds:} The first conscious world serves as a kind of foundation or higher level of reality within which additional worlds or subordinate universes can be created. Now that an initial framework exists (with established laws and an observer present), new fluctuations or acts of creation can occur within it. Simply put, a situation of ``simulation within a simulation'' arises, or universes being born inside a larger Universe.

    \item \textbf{Iterative creation:} The process that occurred in the infinite nothingness can, in a sense, repeat on a smaller scale within the arisen world. The primary observer, and possibly other advanced beings that later develop in the first Universe, might intentionally or accidentally create new worlds; or the mere existence of the first world could naturally lead to the emergence of new structures. In any case, the principle remains the same: a multitude of attempts occur, but only stable configurations persist at the new level. Thus, each generation of worlds begets the next, similar to how within a literary work its own internal story can unfold.

    \item \textbf{Increasing density and structure:} ``Condensation'' means that each successive generation of worlds becomes more concrete or finely defined in its properties. The initial world may have had very general or abstract laws (since it was the first successful one among an endless array of possibilities). The worlds within it can have more finely tuned parameters and greater complexity, but they are also more strictly constrained, as they exist within the context of the parent world’s laws. In other words, the daughter worlds are even more detailed and ``dense'' realities; just as within our physical world the chemical, biological, and social levels of reality possess ever greater complexity and specificity, so in cosmological nesting each new layer adds concreteness and structuredness to existence.

    \item \textbf{Hierarchy of realities:} Reality, according to FZ Theory, has a hierarchical, multi-level structure. At the top is the original, self-sufficient Universe, observed by the primary consciousness. Below it can lie subsequent levels: for example, within the first Universe, advanced beings (or the primary mind itself) might create new universes — whether through experiments, technologies, or even mental practices. These spawned ``universes'' are real to their inhabitants. Each level, on the one hand, obeys the laws of the encompassing world, but on the other hand possesses its own laws (naturally not contradicting the laws of the containing reality). The result is a tree-like or nested structure of being.

    \item \textbf{Continuity of consciousness:} At every new level of world-creation, there is also a corresponding component of consciousness. Either the original observer can be aware of the worlds created within, or within those worlds their own observers arise — either way, the principle holds: the boundaries of any reality are validated by the fact of its perception. Consciousness effectively runs through all levels, linking them with a ``thread of awareness.'' Thus, each condensation of reality is accompanied by a ``condensation'' or increasing complexity of the forms of consciousness.
\end{itemize}

FZ Theory does not specify exactly how many levels of reality may exist or whether this nesting process continues indefinitely. Nevertheless, it follows that the physical Universe we observe might be one such level in a hierarchy — possibly a nested world within a more fundamental reality. Condensation implies that our world is more ``dense'' (rich in structure and strict laws) than the higher, more abstract level, but less dense than any potential worlds that could be created within our own.

It is important to note that although FZ Theory conceptually permits the existence of nested worlds, its primary focus is on the fundamental act of creation — the leap from absolute nothingness to the first ``something.'' The subsequent levels, however intriguing, are an extension of the idea and not essential for understanding the core concept. Nonetheless, recognizing the possible multi-level nature of existence adds depth to the theory and is consistent with the notion of infinite nothingness being able to generate not just one world, but an infinite multitude of worlds at various levels of reality.

\section*{Verification and Confirmation of the Theory}

To bolster the credibility of FZ Theory, a series of computational experiments (unit tests) was designed to simulate its key aspects. Of course, it is impossible to recreate truly infinite nothingness, but these tests approximated the conditions of the theory and checked for consistency with its predictions. Six main tests (denoted FZ unit tests \#1–6) were conducted, each focusing on one important tenet of the theory:

\begin{itemize}
    \item \textbf{Test 1: Spontaneous emergence.} In this test, a large domain of ``nothingness'' was simulated: a virtual space was initialized with no matter or energy, and then random fluctuations were generated within it over many iterations. The result confirmed that with a sufficient number of attempts, a localized structure eventually emerged in the system that could sustain its own existence. During the simulation, after an enormous number of random perturbations, a configuration appeared that did not disintegrate, retaining its structure. This corresponds to the theory’s assertion that a stable world inevitably forms if opportunities for emergence are unlimited (the statistical inevitability of order arising from chaos given infinite trials).

    \item \textbf{Test 2: Stability vs.\ collapse.} This test checked that unstable configurations do indeed quickly break down. By intentionally creating configurations in the model that violated consistency criteria (for example, by introducing contradictory physical laws or nonsensical initial conditions), it was shown that such configurations immediately disintegrated and returned to the initial empty state. This result is in line with the principle that only self-consistent worlds can exist for any length of time, whereas unsound ones vanish without a trace. Nothingness here acts as a natural filter, ``not tolerating'' internally contradictory worlds, which are annulled as swiftly as they appear.

    \item \textbf{Test 3: Necessity of an observer.} To investigate the role of consciousness, the model included the possibility of an ``observer.'' In runs where no element of observation was provided, even some formally stable configurations failed to transition into a fully defined state — they remained blurred, as if indeterminate, or reverted to a chaotic state. However, when at the moment of attempting to form a stable configuration an observational component was introduced (even in a simplified form — for example, a subsystem recording information about the world’s state), the configuration stabilized and acquired definite parameters. This confirms the axiom that the presence of a conscious observer (or at least a mechanism for recording information) is necessary for a world to attain a definite reality. Without an observer, the world in the model seemingly ``doesn’t know'' which state to settle into, and thus easily collapses.

    \item \textbf{Test 4: Emergent consciousness.} The goal of this test was to see if a primitive form of consciousness can spontaneously arise along with a stable structure. Starting with the stable ``world'' obtained in Test 1, the researchers looked for self-organizing patterns within it that could serve the role of an observer. For example, they monitored whether some cyclic process or structure emerged in the stable configuration that responds to the state of the world and thereby ``observes'' it. Remarkably, during the experiment, a self-referential structure did appear inside the stable configuration — it can be interpreted as an elementary ``observer'' formed out of the world’s own structure. Although this nascent mind was extremely primitive, its emergence indicates that once a stable world exists long enough, it is capable of spontaneously generating an element that performs the function of observation. This result is consistent with the theory’s idea that the primary consciousness arises together with the world and can evolve from structures within the world.

    \item \textbf{Test 5: Creation of a nested world.} In this experiment, the stable world from Test 1 (along with the observer component obtained in the previous tests) was taken, and within it new sub-worlds were allowed to form under slightly altered rules (analogous to a ``simulation of a universe within a universe''). The modeling showed that inside the original world, new stable sub-configurations can indeed form given enough time and resources. These ``sub-worlds'' possessed their own self-consistent sets of rules (inherited and modified from the laws of the parent world) and remained stable within the context of the larger universe. This demonstrates the principle of nested reality, at least conceptually: the set of laws of the first world permits the existence within it of autonomous systems with their own specific laws. In essence, the process of a daughter universe emerging within a parent universe was simulated.

    \item \textbf{Test 6: Condensation of laws.} This test compared the characteristics of the sub-worlds with those of the original world. It turned out that the sub-worlds (from Test 5) had more narrow and specific parameters than the parent world. For example, if the laws of the original world were relatively general, the laws of the daughter world proved to be more particular and structured. This reflects the idea of ``condensation'': at each subsequent level, the rules become more detailed, and sub-worlds often evolve toward complexity more rapidly (since they arise already based on the structured system of the higher level). The results of this test are qualitative, but overall they support the assumption that subsequent layers of reality can inherit and refine the laws of the enclosing level, leading to increased complexity and specificity. It was as if a ``compression'' of laws occurred: from the set of universal principles of the higher world, a more rigid set of rules for the inner world emerged.
\end{itemize}

In sum, the experiments show that the main elements of FZ Theory are internally consistent and can, in principle, be realized (at least in simplified form) in a model:

\begin{itemize}
    \item \textbf{Spontaneous emergence of order:} A stable structure can indeed spontaneously arise from chaotic fluctuations (confirming the idea of trial-and-error exploration of possibilities and selection of stable worlds).

    \item \textbf{Factor of observation:} In the absence of an ``observer,'' even stable structures remain indeterminate, whereas in the presence of an observer they acquire a stable, definite state (confirming the necessity of consciousness for concretizing reality).

    \item \textbf{Nested creation:} Within an already existing world, it is possible for new structures to form that possess their own stability (confirming the principle of nested reality and the condensation of laws at lower levels).
\end{itemize}

Of course, these simulations are highly abstract and simplified (since concepts like ``nothingness'' or ``consciousness'' are represented in a rudimentary way). Nonetheless, they serve as a kind of proof-of-concept: they demonstrate that the principles of FZ Theory do not lead to logical contradictions and can be conceptually reproduced in a model experiment. Further substantiation of the theory and comparison with reality will require more sophisticated models and theoretical work, but even these initial tests support confidence in the internal consistency of this conceptual framework.

\section*{Conclusion}

FZ Theory (Theory of Nothing) offers a bold and comprehensive explanation of the first cause: how something (our Universe and reality itself) could emerge from nothing. Relying on the concept of infinite nothingness and the inevitability of random events given an infinite meta-temporal interval, this theory operates without a need for an external creator or pre-existing laws. Instead, the laws and structure of our world appear as the ``winners'' of an infinite cosmic process of trial and error, having passed the filter of stability.

From a scientific perspective, FZ Theory does not conflict with known physics — rather, it seeks to explain how the very foundations of physical laws might have been selected. Once a stable world forms, it evolves according to its internal laws, just as our Universe follows the physical principles we observe. Importantly, the theory functions primarily at a metaphysical or pre-physical level (i.e.\ it describes a stage prior to the emergence of standard physical laws), placing it among speculative concepts. Nonetheless, it remains logically consistent: it posits that any world that successfully emerges observes internal consistency, meaning it does not conflict with itself (even if different possible worlds might differ drastically in their constants and structure).

One advantage of FZ Theory is that it does not require special initial conditions or external triggers: by accepting infinite ``time of nothingness'' as a given (Axiom~1), the theory naturally explains the origin of the Universe without appealing to anything ``before'' it. This approach partially fills a conceptual gap present in some modern cosmological models. For example, eternal inflation scenarios imply the endless generation of ``bubble'' universes, but usually leave the question of where time or the backdrop itself originally came from in the dark. Other proposals (such as the Hartle--Hawking ``no-boundary'' universe) allow that time can emerge from a timeless state, but do not explain why our particular world was realized. FZ Theory provides an elegant bridge over these questions: it introduces infinity as an initial axiom and includes the observer as a necessary element, making the emergence of a specific Universe (ours or any other stable one) the inevitable outcome of an infinite process rather than a contextless fluke.

It is worth noting that positing a non-metric meta-time also offers an explanation for why the physical Universe we observe has a finite age (for example, approximately 13.8 billion years since the Big Bang) and a well-defined arrow of time. According to FZ, our linear time is an emergent artifact of selection: the infinite succession of atemporal attempts is not directly reflected on the timeline of our world, but the result of those attempts was the emergence of a Universe that was born with a ``time arrow'' and a finite past. In other words, inside a stable world, time begins to flow from a particular moment (the act of creation) and in a given direction, reflecting the chosen configuration of laws. Thus, what was an infinite process for the ``external'' nothingness appears to the inhabitants of the emerged world as a history with a beginning and an ordered flow of time. FZ Theory thereby unites the explanation of the origin of time with the explanation of the origin of being.

Admittedly, it is difficult to rigorously model or experimentally verify the process of ``infinite atemporal attempts,'' since current scientific methods assume standard time and repeatability within it. However, the idea of infinite nothingness can be partially illustrated and investigated indirectly. For example, in quantum cosmology there are approaches where the Universe is considered to emerge from a timeless state (the Wheeler--DeWitt equation $H\Psi = 0$ formalizes a condition of ``no time''). In a similar vein, one can imagine the meta-time of nothingness as a superposition of innumerable possibilities, where each potential universe is ``tried out'' in parallel. Likewise, quantum computing and simulation might provide a tool to mimic parallel trials: a quantum computer can represent multiple states simultaneously, and in principle this could serve as a model for exploring how from a superposition of outcomes (analogous to an infinite number of attempts) a single realized world can emerge. Such considerations are still in the realm of hypothesis, but they point to how some aspects of FZ Theory might be made testable or at least vividly demonstrable.

The idea of nested realities and condensation gives the theory an additional dimension, suggesting a potentially infinite nesting or ``tree'' of universes spawning one another. Although this is an extremely intriguing aspect, it remains secondary to the main thesis and requires further theoretical development (in particular, clarifying how exactly laws and properties are transmitted between levels of reality). Nonetheless, the possibility of a multi-level existence is in harmony with the proposed concept of infinite possibilities inherent in nothingness.

In conclusion, FZ Theory (Theory of Nothing) is an original cosmological concept that explains the emergence of existence from nonexistence, the concurrent birth of consciousness with the world, and the possibility of subsequent creation of nested worlds. It represents an independent contribution to cosmological thought, stimulating interest in the discussion and further examination of such ideas. The computational tests conducted lend preliminary support to the theory, showing that its principles can be realized in abstract models and contain no internal contradictions. Going forward, a more detailed mathematical description of ``nothingness,'' exploring connections with existing physical theories (for instance, concepts of the quantum vacuum, emergent time, or inflationary cosmologies), and possibly identifying observable consequences that would provide empirical support for the theory, remain as tasks for future research.

\section*{Appendix: Mathematical Demonstration of the Principle of Infinite Attempts}

For clarity, consider a probabilistic model illustrating the inevitability of a stable world's appearance given an infinite number of attempts (related to Axioms~1--2). Let $p$ be an extremely small but nonzero probability that a single fluctuation will result in a self-consistent stable world (accompanied by consciousness). Then the probability $P(N)$ that at least one stable world will appear in the course of $N$ independent attempts is given by:
\[
P(N) = 1 - (1 - p)^N\,.
\]
Here $(1 - p)^N$ is the probability that none of the $N$ attempts succeeds, and by subtracting it from 1, we obtain the probability of at least one successful outcome. As the number of attempts $N$ increases, this probability rises rapidly. In the limit $N \to \infty$, we get:
\[
\lim_{N \to \infty} P(N) = 1 - \lim_{N \to \infty} (1 - p)^N = 1,\qquad \text{if } p > 0\,.
\]
Thus, even an arbitrarily small but nonzero chance (say $p = 10^{-\text{googol}}$) results in the appearance of at least one stable world becoming virtually guaranteed with an infinite number of trials ($P \to 1$). This simple mathematical fact reflects the essence of FZ Theory: absolute nothingness, with its infinite potential for attempts, inevitably self-generates something. It is as if a paradoxical ``self-saturation'' of nothingness occurs: zero, repeated infinitely, yields a one. This formal result is consistent with Axiom~2 (unlimited possibilities) and Axiom~3 (selection of stable configurations) — nothingness exhaustively explores all variants of being until a viable configuration is found.

As an illustration, consider a numerical example. Let the probability of a successful fluctuation be $p = 10^{-10}$ (this could correspond to an extremely rare quantum tunneling from nothing to something). Then:
\begin{itemize}
    \item For $N = 1$ (one attempt), the probability $P(1) \approx 1 \times 10^{-10}$ (practically zero).
    \item For $N = 10^6$, the probability $P(10^6) \approx 1.15 \times 10^{-4}$ (still an extremely small value, on the order of 0.01\%).
    \item For $N = 10^{12}$, the probability $P(10^{12}) \approx 1 - (1 - 10^{-10})^{10^{12}} \approx 1 - e^{-10^2} \approx 0.9999999999 \approx 1.0$.
\end{itemize}

With $10^{12}$ independent attempts, the event ``emergence of a world'' becomes practically certain (probability $> 99.99999999\%$), and in the limit of an infinite number of attempts, it becomes exactly 1.0. Of course, in reality there is no literal counter of such attempts, but this calculation conceptually models the non-metric infinite ``time of nothingness'': one could say that nothingness had enough opportunities to inevitably realize even an almost impossible event.

For completeness, we also introduce a formal representation of the symmetry of nonexistence and its breaking when something emerges. Absolute nothingness can be treated as a state of complete symmetry (the absence of any distinctions or differentiated states). As the potential of nothingness builds up, this symmetry begins to break. Introduce a dimensionless parameter $\Phi$ characterizing the ``saturation'' of the void with potential (let us say $\Phi = 1$ corresponds to some neutral state, and $\Phi \to \infty$ to the infinite potential of absolute nothingness). Define an asymmetry function:
\[
A(\Phi) = \frac{\Phi^2 - 1}{\Phi^2 + 1}\,.
\]
This quantity measures the deviation from the ideal symmetry between non-being and being. At $\Phi = 1$, we get $A(1) = 0$ (complete symmetry, nothingness is not inclined toward either being or non-being), and as $\Phi \to \infty$, $A(\Phi) \to 1$ (maximal symmetry breaking in favor of the appearance of definite being). For example, at $\Phi \approx 4.38$, the value $A \approx 0.9009$, meaning about 90\% of the symmetry between nonexistence and existence is broken in the direction of existence. In other words, the accumulated potential has already ``tipped the scales'' about 90\% toward the emergence of order. This formal example shows how the growth of the parameter $\Phi$ (reflecting, figuratively speaking, the richness of possibilities or number of trials) leads to an ever-greater bias toward the appearance of differentiation and structure. The process of world creation can be interpreted as a spontaneous symmetry breaking of nonexistence: upon reaching a critical level of potential, nothingness loses its perfect equilibrium and generates a distinction — the first facet of being, a boundary beyond which something appears instead of nothing.

The models and formulas presented above, while simplified, lend a degree of rigor to the framework of FZ Theory. They demonstrate that the idea of infinite attempts mathematically leads to the inevitable emergence of the desired outcome, and they allow for a quantitative discussion of the transition from absolutely symmetric nothingness to the broken symmetry of existence. These derivations do not contradict known principles of probability and symmetry in physics, but rather resonate with them (for example, with ideas about the Universe tunneling from ``nothing'' or about the spontaneous symmetry breaking of the vacuum). In this way, FZ Theory attains not only philosophical cohesion but also a measure of mathematical elegance, showing that infinite nothingness can indeed serve as a legitimate ``generator of being'' when the limit $N \to \infty$ and related physical interpretations are properly taken into account.

\section*{Appendix B: Notation and Definitions}

This appendix collects the main symbols, mathematical quantities, and conceptual variables used throughout the FZ Theory. They are grouped by category and correspond to the formulas and explanations in the main text.

\subsection*{B.1 Core Probabilistic Quantities}

\begin{description}
    \item[$p$] A non-zero probability of a minimal successful fluctuation leading to a stable world.
    \item[$t$] Measure of the number of trials in the ``time of nothingness'' (non-metric meta-time; not chronological physical time).
    \item[$tp$] Trial--probability product; the key parameter of manifestation that controls the saturation of $P$.
    \item[$P(t,p)$] Probability that at least one successful manifestation occurs. Defined as
    \[
    P(t,p) = 1 - e^{-tp}.
    \]
    \item[$P = 1$] Limit of inevitability of manifestation as
    \[
    tp \to \infty \quad \Rightarrow \quad P(t,p) \to 1.
    \]
    \item[$4.60517018$] Critical value of the product $tp$ such that $P = 0.99$. Defined implicitly by solving
    \[
    1 - e^{-tp} = 0.99.
    \]
\end{description}

\subsection*{B.2 Potentiality and Symmetry Quantities}

\begin{description}
    \item[$\Phi$] Potentiality (capacity of nothingness); dimensionless quantity that can grow without bound, $\Phi \to \infty$.
    \item[$A(\Phi)$] Asymmetry function expressing deviation from perfect symmetry between non-being and being. In the model used in this work,
    \[
    A(\Phi) = \frac{\Phi^2 - 1}{\Phi^2 + 1},
    \]
    which is an illustrative phenomenological choice.
    \item[$\Phi = 1$] Symmetric reference point at which $A(1) = 0$ (no bias toward either being or non-being).
    \item[$\Phi \to \infty$] Limit of maximal symmetry breaking, with $A(\Phi) \to 1$, corresponding to a strong bias toward the emergence of definite being.
    \item[$\Phi \approx 4.38$] Example value at which $A(\Phi) \approx 0.9009$, i.e.\ roughly \(90\%\) of the symmetry between nonexistence and existence is broken in favor of existence. (This is conceptually distinct from the critical probability threshold at $tp \approx 4.60517$.)
\end{description}

\subsection*{B.3 Structural Evolution (Densification)}

\begin{description}
    \item[$\rho$] Structural density of a world or proto-world; qualitatively measures how ``dense,'' constrained, or structured a given reality is.
    \item[$\dfrac{d\rho}{dt} = k\rho$] Simple exponential growth law for densification, expressing that the rate of structural growth is proportional to the current density.
    \item[$k$] Growth coefficient (may depend on potentiality level or higher-level constraints).
    \item[$\rho(t) = \rho_0 e^{kt}$] General solution of the densification equation, where $\rho_0$ is the initial structural density at the onset of a given evolutionary phase.
\end{description}

\subsection*{B.4 Stability and Selection}

\begin{description}
    \item[$W$] A world: a set of self-consistent constraints, laws, and structures that jointly define a stable configuration of reality.
    \item[$\sigma$] Spread (variance) of possible world-parameters during the selection process; measures how widely candidate worlds deviate from an attractor.
    \item[$\mu$] Expected (attractor) parameter value for stable worlds; typical or most probable configuration among those that pass the stability filter.
    \item[$\delta$] Width of the stability band around $\mu$; worlds with parameters outside this band tend to be unstable and collapse.
    \item[Gaussian kernel] A Gaussian weighting function used to model the selection of stable worlds from a distribution of candidates, e.g.
    \[
    K(x) = \frac{1}{\sqrt{2\pi}\,\sigma}\,\exp\!\left[-\frac{(x-\mu)^2}{2\sigma^2}\right],
    \]
    where $x$ represents a world-parameter. Configurations with parameters closer to $\mu$ are favored (more stable), while those far from $\mu$ are suppressed.
\end{description}

\subsection*{B.5 Ontological Quantities}

\begin{description}
    \item[$N$] ``Nothingness'': a state with zero distinctions and infinite potentiality; no matter, energy, space, time, or laws.
    \item[$A$] Minimal distinction (first boundary) that breaks the perfect symmetry of $N$ and marks the onset of being.
    \item[$C$] Consciousness (primordial observer) that arises together with the first stable world and defines its boundaries by observation.
    \item[$I$] Information (a difference that makes a difference); structured distinction that shapes form and law.
    \item[$I^\* = F(I^\*)$] Fixed-point consistency condition defining consciousness as a stable informational structure: applying the operator $F$ to $I^\*$ reproduces $I^\*$, expressing self-consistency of the observing system.
\end{description}

\subsection*{B.6 Numerical Constants Used in Code}

\begin{description}
    \item[\texttt{np.expm1}] Numerically stable implementation of $e^{x} - 1$ used to compute $1 - e^{-x}$ without loss of precision for very small $x$.
    \item[\(10^{-20}, 10^{-100}\)] Example small probabilities used in unit tests to probe extreme low-probability regimes.
    \item[\(10^{22}, 10^{102}\)] Example trial magnitudes used to explore asymptotic behaviour as $tp$ becomes very large.
    \item[Precision $= 50$--$100$ digits] Required \texttt{Decimal} precision (number of significant digits) for high-precision stability tests in numerical verification.
\end{description}

\subsection*{B.7 Conceptual Variables (Non-Mathematical)}

\begin{description}
    \item[\emph{Infinite nothing time}] Infinite sequence of non-temporal trials (non-metric meta-time), distinct from physical chronological time.
    \item[\emph{Stability filter}] Conceptual mechanism by which unstable worlds collapse back into nothingness, while self-consistent worlds persist.
    \item[\emph{Nested worlds}] Worlds created inside a parent world, having greater structural density and more specific laws than their encompassing reality.
    \item[\emph{Densification}] Evolution from abstract to concrete, from low-constraint to high-constraint reality; increase in $\rho$ over meta-time.
    \item[\emph{Primordial observer}] First consciousness arising together with the first stable world, providing the reference frame that makes this world definite.
\end{description}


\section*{Acknowledgments}

The author acknowledges the role of personal life experience, which made it possible to synthesize the conceptual components of this work into a unified theoretical framework. No external funding, institutional affiliation, or external collaboration was involved in the development of this study; the FZ Theory and its numerical verification are the result of independent research.

\section*{Data \& Code Availability}

All code used for the computational verification of the FZ Theory (including numerical implementations of the core equations, unit tests, and demonstration scripts) is openly available in the public GitHub repository:

A frozen, citable snapshot corresponding to version~1.0.6 is archived on Zenodo:

\noindent\url{https://doi.org/10.5281/zenodo.17613144}

\noindent\url{https://github.com/DarwinUroborovich/fz-theory.git}

The repository currently contains, in particular:
\begin{itemize}
    \item the core implementation of the main FZ equations in \texttt{src/core.py} (including the manifestation probability $P(t,p)$ and related functions);
    \item a verification suite with unit tests in \texttt{validation/critical\_tests.py}, reproducing the numerical checks discussed in the article;
    \item the full LaTeX manuscript of the theory in \texttt{paper/main.tex};
    \item a demonstration script \texttt{demo.py} that performs example calculations (critical points, extreme cases, high-precision checks) and illustrates how the manifestation probability $P(t,p)$ saturates toward~1 as $t p$ increases;
    \item the file \texttt{requirements.txt} specifying the Python dependencies required to reproduce the numerical experiments;
    \item a detailed \texttt{README.md} with step-by-step instructions for creating a virtual environment, installing dependencies, and running the tests and demo.
\end{itemize}

All materials are released for open examination and may be freely reproduced or extended for further theoretical and numerical research, provided that appropriate credit to the FZ Theory is given.
\end{document}
