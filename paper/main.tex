\title{Theory FZ: Theory of Nothing}
\author{DarwinUroborovich}
\date{2025}

\begin{document}
\maketitle

\begin{abstract}
FZ Theory (also known as the Theory of Nothing) is a conceptual framework that analyzes how a realized world can emerge from what is informally called "nothing". In this work, "nothing" is used in a strictly technical sense as a pre-existential null domain (PND): a fully unstructured condition with no space, time, matter, energy, laws, or degrees of freedom, and with no constraints on what may, in principle, be realized. Within this domain we introduce a non-metric infinite attempt index, representing an unbounded sequence of independent trials rather than physical time.

If the probability that a single trial yields a self-consistent stable configuration is $p > 0$, then the probability that at least one such configuration appears after $t$ attempts is
\[
P(t,p) = 1 - \exp(-t \cdot p),
\]
so that $P \to 1$ as $t \cdot p \to \infty$.

Among all candidate configurations, only stable worlds (self-consistent systems of effective laws and structures) persist, while unstable ones collapse back into nonexistence. Crucially, FZ Theory postulates that a world becomes fully realized only as part of a minimal self-consistent pair (MSCP): a stable configuration together with a minimal observer capable of defining and perceiving its boundaries. This initial observer–world pair provides a primordial framework within which further nested realities can develop, each new level typically acquiring a more concrete and constrained structure (a process termed "condensation" of reality). This work lays out the axiomatic foundations of FZ Theory, describes the mechanism of world formation from a PND, and presents results of computer simulations (six unit tests) that support the internal consistency of the framework. The theory is analyzed for logical coherence and compared with known scientific principles, demonstrating no contradictions with established physical laws while offering a novel, formally articulated perspective on cosmogenesis.
\end{abstract}

\section{Introduction}

The question of how something can arise from nothing has preoccupied philosophers and scientists for centuries. Traditional cosmology usually assumes an initial singularity, a quantum fluctuation, or a primordial vacuum state governed by pre-existing laws. FZ Theory, however, proposes a different starting point: the existence of a pre-existential null domain (PND) — a fully unstructured ontological state with no space, time, matter, energy, or governing principles. Rather than persisting “for an infinitely long duration,” the PND is characterized by an unbounded non-metric attempt index, an infinite sequence of independent trials not tied to physical time. In this domain of absolute non-structure, any admissible configuration may be attempted, regardless of how improbable it would be under conventional physical assumptions.

Instead of invoking external laws or a creator entity, FZ Theory suggests that if the number of attempts within the PND is unbounded, then even extremely improbable events — including the spontaneous emergence of a self-consistent universe — become inevitable in the limit. In an ontological state where no physical properties exist (no energy, no matter, no geometry, no temporal flow), the very absence of constraints provides the arena within which random manifestations can occur. The notion of “infinite time of nothingness” is therefore replaced by the concept of an infinite, lawless combinatorial process, in which all possible proto-configurations are sampled without restriction.

FZ Theory rests on a set of fundamental axioms describing the structure of the PND, the mechanism by which stable configurations are selected, and the conditions under which a potential configuration becomes a realized world. One of the central elements of the framework is the role of consciousness. According to the theory, for a world to become realized — that is, to possess definite boundaries, states, and internal regularities — a minimal form of observation must co-emerge with it. This idea is formalized through the concept of a minimal self-consistent pair (MSCP): a stable world together with a minimal observer capable of distinguishing and registering its features. Reality and consciousness thus appear not as sequential phenomena but as two inseparable aspects of a single generative event.

In the following sections, the axiomatic foundations of the theory are presented in detail. The mechanism by which a stable world emerges from the PND is analyzed, emphasizing the transition from indeterminate potentiality to realized structure. The simultaneous appearance of a minimal observer is discussed as an intrinsic requirement for definitional completeness. The concept of nested creation is introduced, describing how further layers of reality may arise within an already realized world, with each subsequent level exhibiting increased “density” or structural constraint. Next, results are summarized from a series of computational experiments (unit tests) that simulate core aspects of the theory and demonstrate its internal consistency. Finally, the correspondence of FZ Theory with established scientific principles is examined, along with the distinctive conceptual contribution it makes to cosmological thought.

\section*{Axiomatic Foundation}

FZ Theory is based on a set of fundamental axioms defining the properties of what is informally called “nothingness” and the conditions under which a realized world can emerge from it. In the strict formulation adopted in this work, “nothingness” is interpreted as a pre-existential null domain (PND) — a fully unstructured ontological state with no space, no time, no matter, no energy, no fields, no laws, no metric, and no degrees of freedom, and with no constraints on what may, in principle, be realized. Within such a domain, the notion of “duration” is replaced by a non-metric infinite attempt index: an unbounded sequence of abstract, independent trials, not reducible to physical time. The following axioms specify the logical framework of the theory:

\begin{enumerate}
    \item \textbf{Infinite nothingness}  
    It is postulated that there exists an infinite void of absolute non-structure — the PND — in which no matter, energy, spacetime geometry, or physical law is defined. This domain persists through an infinitely extensible sequence of non-metric trials, rather than through temporal duration in the usual sense. “Infinitely long” therefore refers not to linear time, but to an unbounded cardinality of attempts within the PND. In this interpretation, PND functions as a hyper-infinite reservoir of potentiality (formally $\Phi \to \infty$), where the absence of boundaries or constraints ensures that even the rarest logically self-consistent configuration will be realized at least once. Because the number of attempts is unbounded, the probability that any arbitrarily unlikely self-consistent configuration appears approaches 1. Despite having no content, the PND provides an unrestricted background against which potential manifestations can occur.

    \item \textbf{Unlimited possibilities}  
    Within the PND, any logically admissible configuration of proto-laws, proto-structures, or complete “proto-universes” may arise given sufficiently many non-metric attempts. Even if the probability of realizing a particular configuration is extremely small, the unboundedness of the attempt index guarantees that its appearance becomes inevitable in the limit. In effect, the PND cycles through the entire space of possible modes of being — regardless of how large or complex this space may be.

    \item \textbf{Selection by stability}  
    Whenever a proto-universe or candidate configuration arises within the PND, its internal relations, self-consistency, and structural coherence determine whether it is stable or unstable. Stable configurations (i.e., those governed by mutually consistent effective laws that prevent immediate contradiction or collapse) can persist for a significant interval of their internal dynamics, whereas unstable configurations rapidly disintegrate and revert to the PND. Thus, only stable configurations survive for any meaningful duration, while unstable ones vanish without a trace. The PND here acts as a natural selection mechanism, filtering stable worlds from an unbounded multitude of trials.

    \item \textbf{Simultaneous emergence of consciousness}  
    For a stable world to become fully realized, a form of minimal consciousness must arise concurrently with it. This axiom states that a world becomes definite only as part of a minimal self-consistent pair (MSCP): a stable configuration together with a minimal observer capable of discriminating, registering, and thereby defining the world’s boundaries. Without an observer — even a primitive one — a configuration remains indeterminate, not fully instantiated as a realized world. Hence, the first stable world is accompanied by at least one consciousness-like informational process capable of perceiving and stabilizing its characteristics.

    \item \textbf{Primary observer}  
    The minimal consciousness that arises together with the first stable world constitutes a primary observer, not as a supernatural agent but as the informational subsystem necessary to convert an indeterminate configuration into a definite reality. This observer establishes the initial frame of reference, enabling the world to acquire determinate boundaries, states, and regularities. The observer–world pair forms the foundational MSCP upon which all subsequent structure is built.

    \item \textbf{Nested creation (condensation)}  
    Once the primordial world and its primary observer appear, further layers of reality can emerge within it. Each subsequent “generation” of worlds undergoes condensation — acquiring more concrete, specific, and constrained structural laws. In other words, the initial world provides the abstract framework within which increasingly “dense” sub-worlds (nested universes) may originate. Each new level retains consistency with the constraints of the higher-level world but develops more finely tuned internal laws. This iterative cosmogony yields a hierarchy of realities, where the highest level is the most abstract, and progressively lower levels become increasingly concrete.

    \item \textbf{Lawlike emergence}  
    Although the formation of a world within the PND originates from random trials, any configuration that persists necessarily exhibits coherent effective laws. These laws are “selected” from the full space of possibilities as those compatible with stability. Thus, a world that passes the stability filter will appear lawful, ordered, and structured to its inhabitants, even if its initial manifestation was probabilistic. Any world that emerges must conform to self-consistent principles — such as conservation, causality, or symmetry constraints — or else it immediately collapses back into the PND (per Axiom 3).
\end{enumerate}

\subsection*{Summary}

Together, these axioms describe a logically coherent scenario in which something arises from a fully non-structured PND. Random variation generates possibilities; stability filters viable configurations; and consciousness renders them definite. The PND acts as the generator of all potential manifestations, while the remaining axioms determine the properties of the worlds that successfully emerge.

\section{Emergence of a Stable World from Nothingness}

\subsection*{Emergence of a Stable World from Nothingness}

According to FZ Theory, the emergence of a universe (a stable world) is not a singular exceptional event, but an outcome that becomes inevitable under an unbounded sequence of independent attempts. The pre-existential null domain (PND) can be regarded as an unstructured combinatorial substrate in which proto-configurations are continuously sampled in a non-metric manner. Within this framework, random fluctuations correspond to individual attempts to instantiate a coherent configuration capable of sustaining internally consistent structure.

\begin{itemize}
    \item \textbf{Inevitability of fluctuations:}  
    Because the attempt index of the PND is unbounded and the domain imposes no constraints on admissible configurations, even extremely unlikely combinations of parameters will eventually be realized. This parallels a probabilistic process in which, under an infinite number of independent trials, events of arbitrarily small probability become inevitable. Thus, configurations capable of supporting a coherent system of effective laws will appear with certainty in the limit of an unlimited attempt sequence.

    \item \textbf{Multiplicity of attempts:}  
    The emergence of proto-worlds is not restricted to a single location or sequence. Independent attempts may occur across the entire PND without mutual influence, producing numerous short-lived configurations. Most of them do not satisfy the internal consistency criteria and therefore terminate immediately, but a small subset possesses structural features that allow them to persist. In this sense, the PND continuously explores the full space of possible configurations, both sequentially and in parallel.

    \item \textbf{Fate of unstable configurations:}  
    Unstable proto-universes are those whose internal parameter sets yield contradictions or fail to satisfy minimal stability conditions. Such configurations lack the capacity to maintain coherent structure and therefore collapse immediately, reverting to the null domain without generating residual effects. In probabilistic terms, these represent failed trials that do not contribute to the realization of a stable world.

    \item \textbf{Survival of a stable world:}  
    A stable proto-universe, in contrast, contains a self-consistent set of effective laws that permit its continued existence. Once such a configuration appears, it can develop structural organization and maintain persistence independently of the PND. This transition marks the shift from potentiality to realized structure. A stable configuration therefore constitutes the first realized element within the previously structureless domain.
\end{itemize}

The emergence of a stable world can be conceptualized as a selection process operating over an unbounded set of trials. The PND evaluates an effectively infinite space of possibilities, and the appearance of a configuration that satisfies stability criteria constitutes the transition from non-being to being. Although each individual attempt has a negligible probability of success, the unbounded sampling ensures that a stable configuration will eventually arise. Once realized, such a world no longer behaves stochastically but evolves according to its internal laws, which determine its macroscopic development. This perspective resolves the apparent paradox of “something emerging from nothing” by demonstrating that an infinite sequence of null outcomes can lead to a non-null result when the number of trials is unbounded.

\section{The Role of Consciousness}

One of the central components of FZ Theory is the inseparability of the emergence of a stable physical world and the emergence of a corresponding minimal form of observation. This connection addresses a longstanding conceptual question: in what sense can a universe be considered realized if no mechanism exists to register its boundaries, properties, or states?

According to FZ Theory, at the moment a stable world arises, at least one minimal observer co-emerges with it. This is not interpreted as a supernatural act, but as a structural requirement for definitional completeness. The world and the observer together form a minimal self-consistent pair (MSCP), in which the observer performs the function of distinguishing and recording features of the newly realized configuration.

\begin{itemize}
    \item \textbf{Definition of reality’s boundaries:}  
    A minimal observer is necessary for the delimitation of the world’s structural features. Without an observational subsystem, the concepts of space, time, matter, or state lack operational meaning, since no entity exists to register or differentiate them. Observation provides referential grounding, allowing the world’s properties to transition from indeterminate potentiality to determinate structure.

    \item \textbf{Primary conscious being (minimal observer):}  
    The initial observer can be understood as an elementary informational subsystem capable of performing minimal differentiation. It need not resemble biological or human consciousness; it is defined only through its functional role in establishing the world’s boundary conditions. This subsystem arises simultaneously with the stable configuration and does not precede it. Its function is to anchor the world’s properties by continuously registering coherent states.

    \item \textbf{Mutual realization:}  
    The world and the observer arise as mutually dependent components. The stable configuration provides the domain within which observation is possible, while observation ensures that the configuration attains definite values. In this sense, a world without an observer remains an unresolved superposition of potential states, whereas the presence of even the minimal observational subsystem performs the role of selecting, stabilizing, and maintaining definite parameters. This interaction parallels known principles in quantum theory, though FZ Theory applies the concept at a pre-physical ontological level.
\end{itemize}

By incorporating observation into the fundamental process of world formation, FZ Theory replaces the traditional separation between physical states and conscious states with a unified generative framework. Conscious entities appearing later in cosmic evolution—such as biological life—represent elaborations of the minimal observer present at the initial stage. Thus, complex consciousness is not an accidental byproduct but a development of a necessary structural element accompanying the first realized world.

\section{Nested Realities and Condensation}

Once the first stable world has formed — consisting of a self-consistent physical configuration and a primary observer — FZ Theory permits the emergence of additional layers of reality within this initial framework. This process is described through two interrelated concepts: nested realities and the condensation of structure.

\begin{itemize}
    \item \textbf{Nested worlds:}  
    The primordial stable world serves as an upper-level environment within which further autonomous configurations may arise. Because the first world already possesses coherent laws and an observing subsystem, the conditions within it allow new proto-configurations to be instantiated. These subordinate realities may originate through natural dynamical processes or through the deliberate actions of advanced observers emerging within the parent world. The result is a structure in which multiple levels of reality can exist, each being real to the observers inhabiting it.

    \item \textbf{Iterative creation:}  
    The generative mechanism that operated in the pre-existential null domain can reappear on a smaller scale within the formed world. Fluctuation-like processes, computational constructions, or physical interactions may produce new candidate configurations. As in the original creation, many such configurations will be unstable and collapse immediately, while a subset will exhibit internal coherence and persist. Thus, each world can give rise to subsequent worlds by applying the same principles of trial, selection by stability, and persistence.

    \item \textbf{Increasing density and structure:}  
    Condensation refers to the systematic increase in the specificity and constraint of the laws governing each successive level of reality. The initial world may possess broad, minimally defined regularities, as it originated from an unconstrained combinatorial process. Subordinate worlds, however, emerge within an already structured environment and therefore inherit constraints from the parent world. As a result, these daughter worlds may display more narrowly defined parameters and a greater degree of internal organization. This process mirrors the hierarchical structuring observed in physical systems, where each level introduces additional specificity.

    \item \textbf{Hierarchy of realities:}  
    Under this framework, reality assumes a multi-layer hierarchical form. The top level is the primordial world formed directly from the pre-existential null domain, accompanied by the primary observer. Subsequent layers consist of worlds generated within this initial configuration, each governed simultaneously by the laws of the parent world and by its own internally consistent rules. The overall structure resembles a nested hierarchy in which each layer is both autonomous in its internal dynamics and dependent on the constraints of its encompassing level.

    \item \textbf{Continuity of consciousness:}  
    At every level of nested world formation, a corresponding observational component is required to establish the definitional boundaries of the new reality. This observer may be a derivative of the primary consciousness, an emergent informational subsystem within the parent world, or a newly developed entity within the daughter world. In all cases, the presence of an observer ensures that the newly formed world attains definite states. Thus, consciousness is present throughout the hierarchy as a unifying informational thread linking the layers of reality.
\end{itemize}

FZ Theory does not determine whether the hierarchy of nested realities is finite or infinite. It allows for the possibility that the universe we observe represents one level within such a hierarchy. Condensation implies that our world may exhibit more structured and specific laws than the more abstract primordial level, while itself being less dense than any worlds that could emerge within it. Although these nested realities extend the conceptual framework of the theory, the central element remains the original transition from absolute non-being to a realized stable world.

\section{Verification and Confirmation of the Theory}

To assess the internal consistency of FZ Theory, a set of computational experiments was conducted to simulate its fundamental principles. While true pre-existential conditions cannot be reproduced, simplified models can be constructed to evaluate whether the theory’s predictions are coherent. Six conceptual unit tests were designed, each corresponding to a key component of the framework.

\begin{itemize}
    \item \textbf{Test 1: Spontaneous emergence.}  
    A simulated environment representing an unstructured domain was initialized with no imposed laws or organized content. Random perturbations were iteratively introduced. After a sufficiently large number of iterations, a configuration appeared that retained coherence rather than collapsing. This demonstrates the statistical inevitability of stable structures emerging from unbounded trial processes, consistent with the theory’s probabilistic formulation.

    \item \textbf{Test 2: Stability versus collapse.}  
    Configurations deliberately engineered to violate internal consistency (e.g., conflicting rules or mutually incompatible parameters) were tested. Such configurations uniformly collapsed immediately and returned to the unstructured background state. This confirms the principle that only internally coherent structures can persist, while contradictory or unstable configurations cannot endure.

    \item \textbf{Test 3: Necessity of an observer.}  
    The model was extended to include an optional observational subsystem capable of recording information about the system’s states. In runs without such a subsystem, some structurally coherent configurations failed to transition into a determinate state and either remained indeterminate or collapsed. When an observational component was included, the configurations stabilized and acquired definite, persistent parameters. This supports the theoretical assertion that observational functionality is required for definitional completeness.

    \item \textbf{Test 4: Emergent observational subsystem.}  
    Simulations examined whether an informational subsystem capable of performing minimal observational functions could spontaneously arise within a stable world formed in Test 1. Analysis revealed the emergence of a self-organizing structure responding to the state of the system in a manner analogous to observation. Although rudimentary, this confirms that stable environments can generate internal subsystems functionally equivalent to minimal observers.

    \item \textbf{Test 5: Creation of a nested world.}  
    The stable configuration and observer subsystem from earlier tests were used as a basis to allow the formation of subordinate configurations governed by modified or derived rules. These simulations produced new stable sub-worlds possessing their own self-consistent internal rules while operating within the constraints of the parent environment. This confirms the conceptual viability of nested realities.

    \item \textbf{Test 6: Condensation of laws.}  
    Characteristics of the subordinate configurations were compared with those of the original world. Sub-worlds exhibited more narrowly constrained parameters and greater structural specificity. This aligns with the predicted process of condensation, whereby subsequent layers inherit constraints and increase in complexity relative to the primordial world.
\end{itemize}

Taken together, the tests support the conceptual coherence of the core principles of FZ Theory:
\begin{itemize}
    \item spontaneous emergence of order through unbounded trials;
    \item dependence of realization on observational functionality;
    \item feasibility of forming stable subordinate worlds with increased structural constraint.
\end{itemize}

Although these simulations use simplified analogs of “nothingness,” “laws,” and “consciousness,” they demonstrate that the theory’s principles do not produce contradictions within a computational framework. More elaborate models could expand this initial foundation, but the present results provide a consistent proof-of-concept for the theory’s internal architecture.

\section{Conclusion}

FZ Theory (Theory of Nothing) proposes a conceptual framework for the first cause: how a realized universe and, more generally, structured reality could emerge from an initial state of nonexistence. By postulating a pre-existential null domain (PND) with an unbounded non-metric attempt index, the theory replaces the need for external initiating causes or pre-imposed laws with an infinite process of trial and selection. Within this framework, the laws and structure of a given world appear as the outcome of an effectively unbounded exploration of possible configurations, filtered by stability criteria.

From a scientific standpoint, FZ Theory is not intended to modify or replace established physical theories; instead, it aims to provide a conceptual account of how the basic form and parameter choices of those theories might have been selected. Once a stable world emerges, it evolves according to its internal laws, just as the observable Universe follows the principles described by contemporary physics. The theory operates primarily at a pre-physical or metaphysical level, addressing a stage prior to the emergence of standard physical laws. Within this domain it remains logically coherent: any world that successfully arises must be internally consistent, even if different possible worlds would differ drastically in their constants, symmetries, or dynamical structure.

A notable feature of FZ Theory is that it does not require special initial conditions or external triggers. By accepting an unbounded sequence of non-metric attempts (as in Axiom~1), the theory offers a way to explain the origin of a world without appealing to anything ``prior'' to it in a temporal sense. This perspective partially addresses conceptual gaps in some cosmological models. For example, eternal inflation scenarios posit the endless generation of ``bubble'' universes but typically do not specify the origin of the background in which inflation occurs. Other approaches, such as the Hartle--Hawking no-boundary proposal, allow time to emerge from a timeless state, yet they do not directly address why a particular world with specific properties was realized. FZ Theory introduces infinity as a primary axiom and incorporates an observational component as a necessary element, framing the emergence of a specific stable universe as the inevitable outcome of an unbounded selection process.

The introduction of a non-metric attempt index also suggests a way to interpret the finite age and directed arrow of time in the observed Universe. In the FZ framework, linear physical time is an emergent parameter associated with a particular stable configuration. The effectively infinite sequence of pre-physical attempts is not part of the internal temporal axis of the realized world; instead, the realized world begins with an act of selection that establishes an initial moment and a directed evolution. What appears externally as an unbounded pre-history of attempts is internally experienced as a finite past and a well-defined temporal arrow. In this way, the theory links the problem of the origin of time with the problem of the origin of being.

It is challenging to provide direct empirical tests of an ``infinite atemporal attempt'' process, since current scientific methodology assumes standard time and repeatable experiments within it. Nevertheless, elements of the framework can be related to existing ideas in quantum cosmology, where the Universe is sometimes modeled as emerging from a timeless or pre-temporal state (for example, in formulations where the Wheeler--DeWitt equation $H\Psi = 0$ encodes a condition of ``no time''). Analogies can also be drawn with quantum superposition and parallel exploration of states, suggesting that quantum computation and simulation might be used to construct models that emulate aspects of the theory's selection process. Such approaches remain speculative, but they indicate possible pathways to illustrating and probing specific aspects of the framework.

The concepts of nested realities and condensation extend the theory by suggesting that additional layers of structure may arise within an already realized world, potentially generating a hierarchy of worlds with increasing levels of constraint and specificity. While this hierarchy is not essential for understanding the primary transition from nonexistence to existence, it is consistent with the idea that an infinite space of possibilities can encompass multiple levels of realization. Further work would be required to detail the mechanisms by which laws and parameters are transmitted or transformed across different levels of reality.

In summary, FZ Theory (Theory of Nothing) provides a structured cosmological concept that addresses the emergence of existence from nonexistence, the concurrent appearance of a minimal observer with the first stable world, and the possibility of subsequent nested worlds. It is positioned as a conceptual and mathematical framework rather than a fully developed physical theory, but it maintains internal logical consistency and offers a coherent narrative for the selection of worlds. The computational tests discussed in this work serve as proof-of-concept demonstrations, showing that the core principles of the theory can be instantiated in abstract models without generating contradictions. Future research directions include refining the mathematical description of the PND, exploring connections with models of the quantum vacuum, emergent time, and inflationary cosmologies, and identifying potential indirect or analogical signatures that might lend additional support to the framework.

\section*{Appendix: Mathematical Demonstration}

To clarify the probabilistic aspect of FZ Theory, consider a simplified model illustrating the inevitability of the appearance of at least one stable world under an infinite number of attempts (corresponding to Axioms~1 and~2). Let $p$ be an extremely small but nonzero probability that a single fluctuation leads to a self-consistent stable world (with an accompanying minimal observer). The probability $P(N)$ that at least one stable world appears in $N$ independent attempts is

\[
P(N) = 1 - (1 - p)^N.
\]

Here $(1 - p)^N$ is the probability that none of the $N$ attempts succeeds. Subtracting this from $1$ yields the probability of at least one successful outcome. As $N$ increases, $P(N)$ grows monotonically. In the limit $N \to \infty$, we obtain

\[
\lim_{N\to\infty} P(N)
= 1 - \lim_{N\to\infty} (1 - p)^N = 1, \quad \text{for any } p>0.
\]

Thus, even an arbitrarily small but nonzero success probability (for example, $p = 10^{-\mathrm{googol}}$) leads to a probability arbitrarily close to~$1$ for the emergence of at least one stable world when the number of attempts becomes unbounded. This captures the central probabilistic intuition of FZ Theory: an unbounded sequence of attempts in a pre-existential null domain with nonzero potentiality makes the realization of a stable configuration effectively inevitable.

As a numerical illustration, take $p = 10^{-10}$. Then:

\begin{itemize}
    \item For $N = 1$, \\
    $P(1) \approx 1\times 10^{-10}$, effectively zero.
    \item For $N = 10^6$, \\
    $P(10^6) \approx 1.15\times 10^{-4}$, still extremely small.
    \item For $N = 10^{12}$, \\
    \[
    P(10^{12}) \approx 1 - (1 - 10^{-10})^{10^{12}}
    \approx 1 - e^{-10^2}
    \approx 0.9999999999,
    \]
    which is effectively $1.0$ for practical purposes.
\end{itemize}

With $10^{12}$ independent attempts, the emergence of a world becomes practically certain (probability greater than $99.99999999\%$), and in the idealized limit of infinitely many attempts the probability becomes exactly~1. In the FZ context, there is no literal counter recording these attempts; instead, this calculation serves as an analogy for the non-metric infinite attempt index: one can say that the PND has ``enough opportunities'' to realize even events with extremely low individual probability.

To complement this probabilistic picture, we introduce a simple phenomenological model of symmetry breaking between nonexistence and existence. Absolute nothingness can be idealized as a state of complete symmetry, in which no distinctions or differentiated states are present. As potentiality accumulates, this symmetry begins to shift. Introduce a dimensionless parameter $\Phi$ that represents the level of potentiality (for example, $\Phi = 1$ as a reference state and $\Phi \to \infty$ as the limit of maximal potential). Define an asymmetry function

\[
A(\Phi) = \frac{\Phi^2 - 1}{\Phi^2 + 1}.
\]

This function quantifies the deviation from perfect symmetry between non-being and being. At $\Phi = 1$, we have $A(1) = 0$, corresponding to a symmetric reference state with no bias toward either nonexistence or existence. As $\Phi \to \infty$, $A(\Phi) \to 1$, representing maximal asymmetry in favor of realized being. For instance, at $\Phi \approx 4.38$, $A(\Phi) \approx 0.9009$, indicating that approximately $90\%$ of the symmetry has been broken toward existence. In this heuristic model, increasing $\Phi$ can be interpreted as increasing the effective richness of possibilities or the effective number of attempts, leading to an increasing bias toward the emergence of structured configurations.

This construction is not proposed as a fundamental physical law but as an illustrative mathematical mapping that makes the notion of ``self-saturation'' of the PND more concrete. Here, ``symmetry,'' ``asymmetry,'' and the values $A(\Phi)\in[0,1]$ are not physical quantities but conceptual parameters of the model, introduced solely as an intuitive way to represent how increasing potentiality biases the system toward realized being.

It shows how a parameter representing accumulated potentiality can be associated with a monotonic increase in the preference for structured being over undifferentiated nonexistence. The transition from a symmetric state (no realized worlds) to an asymmetric state (at least one realized world) can then be described as a form of spontaneous symmetry breaking at the level of ontological possibilities.

The models and formulas presented here are simplified and phenomenological, but they provide a transparent way to link the qualitative ideas of FZ Theory with explicit mathematical expressions. They demonstrate that the assumption of an unbounded attempt process with nonzero success probability naturally leads to the inevitability of at least one successful realization, and they support a quantitative discussion of how an initially symmetric state can evolve into one where structured existence is favored. These constructions are consistent with established probability theory and with general ideas about symmetry breaking in physics, while remaining explicitly interpretive rather than derived from a specific dynamical model. In this sense, they add a measure of mathematical structure to the theory’s conceptual core and help clarify how an ``infinite nothingness'' can function as a generative backdrop for realized worlds.


\section*{Appendix B: Notation and Definitions (Selected Quantities)}

\subsection*{B.2 Potentiality and Symmetry Quantities}

\begin{itemize}
    \item \textbf{$\Phi$ (Phi):}  
    Potentiality or capacity of the pre-existential null domain (PND); a dimensionless quantity that can grow without bound ($\Phi \to \infty$). It represents, in an abstract way, the “amount” of available possibility.

    \item \textbf{$A(\Phi)$:}  
    Asymmetry function expressing the deviation from perfect symmetry between non-being and being. In the illustrative model used in this work, it is defined as  
    \[
    A(\Phi) = \frac{\Phi^2 - 1}{\Phi^2 + 1}.
    \]  
    This choice is phenomenological: it provides a smooth, monotonic mapping from $\Phi$ to a measure of asymmetry, but it is not derived from a specific dynamical law.

    \item \textbf{$\Phi = 1$:}  
    Reference point at which $A(1) = 0$, corresponding to a symmetric state with no bias toward either nonexistence or existence.

    \item \textbf{$\Phi \to \infty$:}  
    Limit of maximal symmetry breaking, with $A(\Phi) \to 1$, corresponding to a strong bias toward the emergence of definite being.

    \item \textbf{$\Phi \approx 4.38$:}  
    Example value at which $A(\Phi) \approx 0.9009$, meaning that roughly 90 percent of the symmetry between nonexistence and existence is broken in favor of existence. This example is conceptually distinct from the critical probability threshold at $t p \approx 4.60517$ used in the probabilistic model.
\end{itemize}

\subsection*{B.3 Structural Evolution (Densification)}

\begin{itemize}
    \item \textbf{$\rho$ (rho):}  
    Structural density of a world or proto-world; a qualitative measure of how “dense,” constrained, or structured a given reality is. Higher $\rho$ corresponds to more detailed and restrictive laws or configurations.

    \item \textbf{$\frac{d\rho}{dt} = k\rho$:}  
    Simple exponential growth model for densification, expressing that the rate of structural growth is proportional to the current structural density $\rho$. Here $t$ represents an appropriate parameter of evolution, and $k$ is a growth coefficient.

    \item \textbf{$k$:}  
    Growth coefficient that may depend on the level of potentiality or on constraints imposed by higher-level structures. It sets the characteristic rate at which structure accumulates.

    \item \textbf{$\rho(t) = \rho_0 e^{k t}$:}  
    General solution of the densification equation, where $\rho_0$ is the initial structural density at the beginning of a given evolutionary phase. This model is illustrative and is used to describe the idea that structure can grow exponentially under suitable conditions.
\end{itemize}

\subsection*{B.4 Stability and Selection}

\begin{itemize}
    \item \textbf{$W$:}  
    A world: a set of self-consistent constraints, laws, and structures that jointly define a stable configuration of reality.

    \item \textbf{$\sigma$ (sigma):}  
    Spread (variance) of possible world-parameters during the selection process; measures how widely candidate worlds deviate from a preferred or stable value.

    \item \textbf{$\mu$ (mu):}  
    Expected or attractor parameter value for stable worlds; represents a typical or most probable configuration among those that satisfy stability criteria.

    \item \textbf{$\delta$ (delta):}  
    Width of the stability band around $\mu$; configurations with parameters outside this band tend to be unstable and collapse back into the PND.

    \item \textbf{Gaussian kernel:}  
    A Gaussian weighting function used to model the selection of stable worlds from a distribution of candidates, typically of the form  
    \[
    K(x) \propto \exp\!\left[-\frac{(x - \mu)^2}{2\sigma^2}\right],
    \]  
    where $x$ denotes a world-parameter. Configurations closer to $\mu$ are favored (more likely to be stable), while those far from $\mu$ are suppressed.
\end{itemize}

\subsection*{B.5 Ontological Quantities}

\begin{itemize}
    \item \textbf{$N$:}  
    “Nothingness” understood as a state with zero distinctions and infinite potentiality: no matter, no energy, no space, no time, and no explicit laws.

    \item \textbf{$A$ (as a concept, not $A(\Phi)$):}  
    Minimal distinction or first boundary that breaks the perfect symmetry of $N$ and marks the onset of being. It can be interpreted as the first stable differentiation within the PND.

    \item \textbf{$C$:}  
    Consciousness or primordial observer that arises together with the first stable world and defines its boundaries through observation. $C$ plays the role of the observational component in the minimal self-consistent pair (MSCP).

    \item \textbf{$I$:}  
    Information, understood as a “difference that makes a difference”; structured distinction that shapes form, law, and state within a world.

    \item \textbf{$I^{*} = F(I^{*})$:}  
    Fixed-point condition defining consciousness as a stable informational structure. Here $I^{*}$ denotes an informational configuration, and $F$ is an operator that maps informational states to informational states. The equality $I^{*} = F(I^{*})$ expresses self-consistency of the observing system: its structure is preserved under its own update dynamics.
\end{itemize}

\subsection*{B.6 Numerical Constants Used in Code}

\begin{itemize}
    \item \textbf{\texttt{np.expm1}:}  
    Numerically stable implementation of the function $\exp(x) - 1$, used in the code to compute quantities like $1 - \exp(-x)$ for very small $x$ without losing precision.

    \item \textbf{$10^{-20}$, $10^{-100}$:}  
    Example small probabilities used in unit tests to probe extreme low-probability regimes and to verify that the numerical implementation behaves correctly under such conditions.

    \item \textbf{$10^{22}$, $10^{102}$:}  
    Example large trial counts used to explore asymptotic behavior as the product $t p$ becomes very large, verifying that probabilities saturate as expected.

    \item \textbf{Precision = 50--100 digits:}  
    Required decimal precision for high-precision tests in the numerical verification of the theory’s formulas. This precision is used in the \texttt{Decimal}-based computations to avoid round-off errors in extreme parameter regimes.
\end{itemize}

\subsection*{B.7 Conceptual Variables (Non-Mathematical)}

\begin{itemize}
    \item \textbf{Infinite nothing time:}  
    Conceptual shorthand for an infinite sequence of non-temporal trials (non-metric meta-time), distinct from physical chronological time. It represents the unbounded attempt index of the PND.

    \item \textbf{Stability filter:}  
    Conceptual mechanism by which unstable worlds collapse back into nothingness, while self-consistent worlds persist. In practice, this corresponds to discarding inconsistent configurations in the selection process.

    \item \textbf{Nested worlds:}  
    Worlds created inside a parent world, having greater structural density and more specific laws than their encompassing reality. They represent lower levels in the hierarchy of realities described by FZ Theory.

    \item \textbf{Densification:}  
    Evolution from abstract to concrete, from low-constraint to high-constraint reality; associated with an increase in structural density $\rho$ over meta-time or over appropriate evolutionary parameters.

    \item \textbf{Primordial observer:}  
    First minimal form of consciousness arising together with the first stable world, providing the reference frame that makes this world definite. It constitutes the observational component of the initial minimal self-consistent pair (MSCP).
\end{itemize}

\section*{Acknowledgments}

The author acknowledges the role of personal life experience, which made it possible to synthesize the conceptual components of this work into a unified theoretical framework. No external funding, institutional affiliation, or external collaboration was involved in the development of this study; the FZ Theory and its numerical verification are the result of independent research.

\section*{Data \& Code Availability}

All code used for the computational verification of the FZ Theory (including numerical implementations of the core equations, unit tests, and demonstration scripts) is openly available in the public GitHub repository.

A frozen, citable snapshot corresponding to version~1.0.6 is archived on Zenodo:

\noindent\url{https://doi.org/10.5281/zenodo.17613144}

\noindent\url{https://github.com/DarwinUroborovich/fz-theory.git}

The repository currently contains, in particular:
\begin{itemize}
    \item the core implementation of the main FZ equations in \texttt{src/core.py} (including the manifestation probability $P(t,p)$ and related functions);
    \item a verification suite with unit tests in \texttt{validation/critical\_tests.py}, reproducing the numerical checks discussed in the article;
    \item the full \LaTeX{} manuscript of the theory in \texttt{paper/main.tex};
    \item a demonstration script \texttt{demo.py} that performs example calculations (critical points, extreme cases, high-precision checks) and illustrates how the manifestation probability $P(t,p)$ saturates toward~1 as $t p$ increases;
    \item the file \texttt{requirements.txt} specifying the Python dependencies required to reproduce the numerical experiments;
    \item a detailed \texttt{README.md} with step-by-step instructions for creating a virtual environment, installing dependencies, and running the tests and demo.
\end{itemize}

All materials are released for open examination and may be freely reproduced or extended for further theoretical and numerical research, provided that appropriate credit to the FZ Theory is given.

\end{document}
